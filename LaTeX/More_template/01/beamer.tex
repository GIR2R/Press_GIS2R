%\documentclass[compress]{beamer}
\documentclass[xcolor=dvipsnames]{beamer}

%%%%%%%%%%%%%%%%%%%%%%%%%%%%%% para escribir en espa�ol
    \usepackage[spanish]{babel}
    \usepackage[ansinew]{inputenc}
    \usepackage{ragged2e}

%%%%%%%%%%%%%%%%%%%%%%%%%%%%%% caracter�sticas de Beamer
    \usetheme{Warsaw} % Beamer Theme
    \useoutertheme[subsection=false]{smoothbars} % Beamer Outer Theme
    \useinnertheme{rectangles}

%%%%%%%%%%%%%%%%%%%%%%%%%%%%%% definimos colores
    \definecolor{micolor}{RGB}{98,137,39}
    \setbeamercolor*{palette primary}{use=structure,fg=white,bg=micolor}
    \setbeamercolor*{palette secondary}{use=structure,fg=white,bg=micolor!75!black}
    \setbeamercolor*{palette tertiary}{use=structure,fg=white,bg=micolor!50!black}
    \setbeamercolor*{palette sidebar primary}{use=structure,fg=micolor!10}
    \setbeamercolor*{palette sidebar tertiary}{use=structure,fg=micolor!50}

%%%%%%%%%%%%%%%%%%%%%%%%%%% color de Beamer
    \usecolortheme[named=Brown]{structure}

%%%%%%%%%%%%%%%%%%%%%%%%%%%%%% t�tulo, autor, etc...
    \title[T�tulo corto]{T�tulo largo}
    \subtitle{Subt�tulo}
    \author[Autor corto]
    {%
        Autor 1\inst{1}\\[-2mm]
        \texttt{\tiny kk@kk.es}\\
        \and%
        Autor 2\inst{2}\\[-2mm]
        \texttt{\tiny kk@kk.com}\\
    }
    \institute[Instituci�n corta]
    {%
          \inst{1}%
            Profesor\vspace{-2mm}
          \and
          \inst{2}%
            Profesora\vspace{-2mm}
    }
    \date[Fecha corta]{Fecha larga}
    \subject{Generaci�n de presentaciones}

%\pgfdeclareimage[height=0.8cm]{university-logo}{logo}
%\logo{\pgfuseimage{university-logo}}

\begin{document}

\begin{frame}
    \titlepage
\end{frame}

\begin{frame}
    \frametitle{�ndice}

    \tableofcontents
\end{frame}

\section{Instalaci�n de \LaTeX}

\begin{frame}
    \frametitle{Instalaci�n de \LaTeX}

    Los pasos a seguir est�n en \href{http://unrinconparalatex.blogspot.com.es/}{http://unrinconparalatex.blogspot.com.es/}.
\end{frame}

\section{Empezamos con Beamer}

\subsection{Tipos de bloques}

\begin{frame}
    \frametitle{Primeros pasos: bloques}

    \begin{block}{T�tulo 1}
        Entorno \textit{block}
    \end{block}

    \begin{exampleblock}{T�tulo 2}
        Entorno \textit{exampleblock}
    \end{exampleblock}

    \begin{alertblock}{T�tulo 3}
        Entorno \textit{alertblock}
    \end{alertblock}

\end{frame}

\subsection{Usando columnas}

\begin{frame}
    \frametitle{Entorno \textit{columns}}

    Un ejemplo:

    \begin{columns}
        \column{.5\textwidth}
        \framebox[\textwidth]{
            Contenido de la 1� columna.
        }
        \column{.5\textwidth}
        \framebox[\textwidth]{
            Contenido de la 2� columna.
        }
    \end{columns}

    \vspace{0.2cm}

    Otro ejemplo:

    \begin{columns}
        \begin{column}{0.5\textwidth}
            Contenido de la 1� columna.
        \end{column}
        \hfill
        \begin{column}{0.5\textwidth}
            Contenido de la 2� columna.
        \end{column}
    \end{columns}

    \vspace{0.2cm}

    Y otro:

    \begin{columns}
        \begin{column}{0.5\textwidth}
            \begin{block}{}
                Contenido de la 1� columna.
            \end{block}
        \end{column}
        \hfill
        \begin{column}{0.5\textwidth}
            \begin{block}{}
                Contenido de la 2� columna.
            \end{block}
        \end{column}
    \end{columns}

    \vspace{0.2cm}

    Y el �ltimo:

    \begin{block}{}
        \begin{columns}
            \begin{column}{0.42\textwidth}
                Contenido de la 1� columna.
            \end{column}
            \hfill
            \begin{column}{0.42\textwidth}
                Contenido de la 2� columna.
            \end{column}
        \end{columns}
    \end{block}
\end{frame}

\section{Efectos especiales}

\subsection{Pausas}

\begin{frame}[fragile]
    \frametitle{\textit{pause} y algo m�s}

    \begin{itemize}
        \item Uno \pause
        \item Dos \pause
        \item y Tres. \pause
    \end{itemize}

    \begin{itemize}[<+-| alert@+>]
        \item Cuatro.
        \item Cinco.
        \item Seis.
        \item Siete.
    \end{itemize}

    \begin{itemize}
        \item<8-> Ocho.
        \item<11> Once.
        \item<9-10> Nueve, diez y no once.
    \end{itemize}
\end{frame}

\subsection{Tapamos y destapamos}

\begin{frame}
    \frametitle{\textit{only}, \textit{onslide}, \textit{visible}}

    \only<1>{
        \begin{block}{}
            S�lo en el primer click
        \end{block}
    }

    \onslide<2-3>{
        \begin{exampleblock}{}
            En los clicks 2 y 3.
        \end{exampleblock}
    }

    \visible<2>{
        \begin{alertblock}{}
            Visible en el segundo click.
        \end{alertblock}
    }
    
    \uncover<4>{
        \begin{alertblock}{}
            Descubierto en el cuarto click.
        \end{alertblock}
    }
\end{frame}

\section{Justificar texto}

\begin{frame}[fragile]\justifying
    \begin{alertblock}{���OJO!!!}
        Esta transparencia no tiene t�tulo!!!
    \end{alertblock}

    \vspace{0.5cm}

    Para justificar el texto hay que especificar  \verb|\usepackage{ragged2e}| en el pre�mbulo y a continuaci�n del \textit{frame} o caja (\textit{block}, \textit{exampleblock}, etc) escribir \verb|\justifying|:

    \begin{block}{Ejemplo de texto justificado}\justifying
        En un lugar de la Mancha, de cuyo nombre no quiero acordarme, no ha mucho tiempo que viv�a un hidalgo de los de lanza en astillero, adarga antigua, roc�n flaco y galgo corredor. Una olla de algo m�s vaca que carnero, salpic�n las m�s noches, duelos y quebrantos los s�bados, lentejas los viernes, alg�n palomino de a�adidura los domingos, consum�an las tres partes de su hacienda...
    \end{block}

\end{frame}

\section{Manual de ayuda}

\begin{frame}
    \frametitle{Referencias}

    Queda mucho por aprender, todo est� en: % \cite{manual}.

    \begin{thebibliography}{10}
        \bibitem{manual} The BEAMER \emph{class}. \href{ftp://ftp.dante.de/tex-archive/macros/latex/contrib/beamer/doc/beameruserguide.pdf}{User Guide for version 3.33}.
    \end{thebibliography}
\end{frame}

\end{document} 