\documentclass[10pt]{beamer}%
\usetheme[%
%width=2cm, %define el ancho de la barra de navegación
%left, %barra navegacion a la izquierda
%right %barra navegacion a la derecha
]{Aalborg}%
%========================================%
%Para cambiar algunos colores en el tema %
%pueden descomentar y compilar para ver  %
%los cambios, ojo que lo pueden persona- %
%lizar modificando el porcentaje de color%
%\setbeamercolor{Aalborg}{fg=red!20,bg=red}%
%\setbeamercolor{sidebar}{bg=red!20}%
%\setbeamercolor{structure}{fg=red}%
%\setbeamercolor{frametitle}{fg=blue}%
%\setbeamercolor{normal text}{bg=gray!10}%
%========================================%
\graphicspath{{Images/}{Images/logo/}}%
\usepackage{geometry}%
\geometry{%
  paperwidth=1.33\paperwidth,%
  paperheight=\the\paperheight,%
  vmargin=0cm,%
}
\usepackage[utf8]{inputenc}%
\usepackage[spanish]{babel}%
\usepackage[T1]{fontenc}%
\usepackage{ragged2e}%
\usepackage{helvet}%
\newcommand{\chref}[2]{%
  \href{#1}{{\usebeamercolor[bg]{Aalborg}#2}}%
}%
\AtBeginSection[]{
	\begin{frame}
	\vfill
	\centering
	\begin{beamercolorbox}[sep=8pt,center,shadow=true,rounded=true]{title}
		\usebeamerfont{title}\insertsectionhead\par%	
	\end{beamercolorbox}		
	\end{frame}
}
%%%%%%%%%%%%%%%%%%%%%%%%%%%%%%%%%%%%%%%%%%%%%%
%==========================%
%Titulo de la presentacion %
\title[\textbf{Titulo corto}]%
{{{\textcolor{white} \Huge Titulo largo de la presentación}}}%
%==========================%
\subtitle{Charlas del Grupo GIS2R}  % could also be a conference name
\author[Luis Robles]%nombre corto (al pie del logo)
       {Luis Robles}%nombre largo (opcional, al inicio en el titulo)
%=================================%
%En el caso de ser varios autores %
%  \author[Luis Robles] % optional, use only with lots of authors
% {
% Luis Alberto Robles
% }
%=================================%
%==========================================================================%
%Ingresa solo el nombre de la institucion a la que pertenece (affiliation) %
% \institute[LabTel \\ FCF - UNMSM]{
% Laboratorio de Teledetección - LabTel \\
% Universidad Nacional Mayor de San Marcos}
%==========================================================================%
%=========================================================================%
%Ingresar nombre y logo de la institucion a la que pertenece (affiliation)%
\institute[
{\includegraphics[scale=0.05]{Images/logo/unmsm}}\\ %logo insti
LabTel \\ %laboratorio
FCF - UNMSM %facultad y universidad
]%
{%
Laboratorio de Teledetección - LabTel\\
Facultad de Ciencias Físicas - UNMSM%
}%
%=========================================================================%
%======%
%Fecha %
\date{\today}
%======%
%=================================%
%Logo de caratula inicio de press %
\pgfdeclareimage[height=2.5cm]{titlepagelogo}{Images/logo/gis2r}
\titlegraphic{%
  \pgfuseimage{titlepagelogo}
}
%=================================%
%=================================%
%Logo de esquina superior derecha %
\pgfdeclareimage[height=1.6cm]{mainlogo}{Images/logo/gis2r}
\logo{\pgfuseimage{mainlogo}}
%=================================%

\begin{document}
\justifying
%%%%%%%%%%%%%%%%%%%%%%%%%%%%%%%%%%%%%%%%%%%%%%%%
%%%%%%%%%%%%%%%%%%%%%%%%%%%%%%%%%%%%%%%%%%%%%%%%
% the titlepage
{\aauwavesbg
\begin{frame}[plain,noframenumbering] % the plain option removes the sidebar and header from the title page
  \titlepage
\end{frame}}
%%%%%%%%%%%%%%%%%%%%%%%%%%%%%%%%%%%%%%%%%%%%%%%%
%%%%%%%%%%%%%%%%%%%%%%%%%%%%%%%%%%%%%%%%%%%%%%%%
% TOC
\begin{frame}{Contenido}{}
\begin{footnotesize}
	\tableofcontents	
\end{footnotesize}
\end{frame}
%<================<*>================>%
%<================<*>================>%
%en el caso que quiera separar por partes
%\part{Introducción}
%\begin{frame}{}
% \partpage
%\end{frame}
%<================<*>================>%
%<================<*>================>%
\section{Introducción}
\begin{frame}{Introducción}
\begin{block}{Requerimientos}
		\begin{itemize}
			\item Instalar compilador, Recomiendo Texlive
			\item Instalar un editor los listados funcionan enLinux y windows:
				\begin{itemize}
					\item Texmaker \url{http://www.xm1math.net/texmaker/}
					\item Texworks \url{https://www.tug.org/texworks/}
					\item Kile \url{http://kile.sourceforge.net/}
					\item Sublime \url{https://www.sublimetext.com/}
				\end{itemize}
			\item Solo por cuestiones de gustos no uso Miktex (\url{http://miktex.org/}), pero se que en el entorno windows es muy usado 
			\item Trabajen con el que se sientan más cómodos, en lo personal uso Sublime, pero para iniciar los tres primeros son perfectos.
		\end{itemize}
\end{block}
\end{frame}
%<================<*>================>%
%<================<*>================>%
\section{Usando \LaTeX} % (fold)
\begin{frame}{Ingresando texto mediante minipages}
	\begin{minipage}{0.48\textwidth}
			Lorem ipsum dolor sit amet, consectetur adipisicing elit. Eius ipsa harum, amet beatae quasi eligendi quae porro, quas commodi veritatis!
	\end{minipage}
	\hfill
	\begin{minipage}{0.48\textwidth}
			Lorem ipsum dolor sit amet, consectetur adipisicing elit. Deleniti, et quam debitis enim, harum ipsam doloribus. Temporibus blanditiis doloribus recusandae porro repellendus optio, veniam eaque, voluptates quibusdam quae. Delectus, cumque.
	\end{minipage}

	\vspace{0.5cm}

	Se pueden ingresar más minipages, considerando las dimensiones de cada. Si es muy grande el ultimo minipage corre a la siguiente fila
\end{frame}
%<================<*>================>%
%<================<*>================>%
\section{Imágenes} % (fold)
\begin{frame}{Ingresando imágenes}
	\begin{figure}[!h]
		\centering
		\includegraphics[scale=0.2]{gis2r}
		\caption{Descripción de imagen}
		\label{fig:gis2r}
	\end{figure}
\end{frame}
%<================<*>================>%
%<================<*>================>%
\begin{frame}{Ingresado imágenes en dos minipages}	
	\begin{minipage}{0.49\textwidth}
				\begin{figure}[!h]
				\centering
				\includegraphics[scale=0.1]{gis2r}
				\caption{Grupo GIS2R}
				\label{fig:gis2r}
			\end{figure}
	\end{minipage}
	\begin{minipage}{0.49\textwidth}
				\begin{figure}[!h]
				\centering
				\includegraphics[scale=0.1,angle=45]{gis2r}
				\caption{Grupo GIS2R, rotado}
				\label{fig:gis2r}
			\end{figure}
	\end{minipage}
\end{frame}
%<================<*>================>%
%<================<*>================>%
\begin{frame}{Ingresado imágenes en dos minipages}	
	\begin{minipage}{0.49\textwidth}
				\begin{figure}[!h]
				\centering
				\includegraphics[scale=0.1]{gis2r}
				\caption{Grupo GIS2R}
				\label{fig:gis2r}
			\end{figure}
	\end{minipage}
	\begin{minipage}{0.49\textwidth}
				\begin{figure}[!h]
				\centering
				\includegraphics[scale=0.1,angle=180]{gis2r}
				\caption{Grupo GIS2R, rotado}
				\label{fig:gis2r}
			\end{figure}
	\end{minipage}

	\begin{minipage}{0.49\textwidth}
				\begin{figure}[!h]
				\centering
				\includegraphics[scale=0.1,angle=180]{gis2r}
				\caption{Grupo GIS2R}
				\label{fig:gis2r}
			\end{figure}
	\end{minipage}
	\begin{minipage}{0.49\textwidth}
				\begin{figure}[!h]
				\centering
				\includegraphics[scale=0.1]{gis2r}
				\caption{Grupo GIS2R, rotado}
				\label{fig:gis2r}
			\end{figure}
	\end{minipage}
\end{frame}
%<================<*>================>%
%<================<*>================>%
\section{Block} % (fold)
\begin{frame}{Usando block}
	\begin{block}{Titulo del block 1}
			Lorem ipsum dolor sit amet, consectetur adipisicing elit. Minima, possimus minus recusandae atque fugit esse.
	\end{block}
	\begin{block}{Titulo del block 2}
			Lorem ipsum dolor sit amet, consectetur adipisicing elit. Minima, possimus minus recusandae atque fugit esse.
	\end{block}
\end{frame}
%<================<*>================>%
%<================<*>================>%
\begin{frame}{Usando block dentro del minipage}
	\begin{minipage}{0.49\textwidth}
	\begin{block}{Titulo del block 1}
			Lorem ipsum dolor sit amet, consectetur adipisicing elit. Minima, possimus minus recusandae atque fugit esse.
	\end{block}
	\end{minipage}
	\hfil
	\begin{minipage}{0.49\textwidth}
	\begin{block}{Titulo del block 2}
			Lorem ipsum dolor sit amet, consectetur adipisicing elit. Minima, possimus minus recusandae atque fugit esse.
	\end{block}
	\end{minipage}
\end{frame}
%<================<*>================>%
%<================<*>================>%
\begin{frame}{Color en titulo de block}
\begin{block}{Observation 1}
Simmons Hall is composed of metal and concrete.
\end{block}
\begin{exampleblock}{Observation 2}
Simmons Dormitory is composed of brick.
\end{exampleblock}
\begin{alertblock}{Conclusion}
Simmons Hall $\not=$ Simmons Dormitory.
\end{alertblock}
\end{frame}
%<================<*>================>%
%<================<*>================>%
\section{Código}
\begin{frame}{Ingresando código}
	\begin{center}
		{\Huge Falta :D}
	\end{center}
\end{frame}
%<================<*>================>%
%<================<*>================>%
{\aauwavesbg%
\begin{frame}[plain,noframenumbering]%
  \finalpage{Gracias por su atención\\
  ¿preguntas?}
\end{frame}}
\end{document}